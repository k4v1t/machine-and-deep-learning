\documentclass[11pt]{article}
% -------------------------------
% Shared packages and commands
% -------------------------------

% --- Page size & margins ---
\usepackage[a4paper,top=2cm,bottom=2.5cm,left=2cm,right=2cm]{geometry}

% --- Title formatting ---
\usepackage{titling}

% Move the whole title block up
\setlength{\droptitle}{-2em}

% Keep author and date on their own lines, just reduce the gaps
\pretitle{\begin{center}\LARGE\bfseries}%
\posttitle{\par\end{center}\vspace{-0.5em}} % space after title
\preauthor{\begin{center}\large}%
\postauthor{\par\end{center}\vspace{-0.5em}} % space after author
\predate{\begin{center}\small}%
\postdate{\par\end{center}\vspace{-1em}} % space after date block


% --- Math & theorem environments ---
\usepackage{amsmath, amssymb, amsthm}

% --- Graphics & tables ---
\usepackage{graphicx}
\usepackage{booktabs}

% --- Hyperlinks ---
\usepackage{hyperref}
\hypersetup{
  colorlinks=true,
  linkcolor=blue,
  urlcolor=blue
}

% --- Optional: underlined hyperlinks (remove if not desired) ---
% \usepackage[normalem]{ulem}
% \renewcommand\UrlFont{\uline}

% --- Custom macros (add your own below) ---
% \newcommand{\R}{\mathbb{R}}

\title{SPC707P Machine and Deep Learning — Week 01 Project}
\author{Kavit Tolia} 
\date{\today}

\begin{document}
\maketitle

\section{Pick 5 of your favourite datasets}
I have picked the following 5 datasets for this week's project: \\
1. Adult Income or Census Income from US Census Bureau: \href{https://archive.ics.uci.edu/dataset/2/adult}{\underline{Adult Income}} \\
2. Air Quality or AirQualityUCI using sensors in Italy: \href{https://archive.ics.uci.edu/dataset/360/air+quality}{\underline{Air Quality}} \\
3. Micro Gas Turbine Electrical Energy Prediction: \href{https://archive.ics.uci.edu/dataset/994/micro+gas+turbine+electrical+energy+prediction}{\underline{Electrical Energy Prediction}} \\
4. Heart Disease (Cleveland): \href{https://archive.ics.uci.edu/dataset/45/heart+disease}{\underline{Heart Disease}} \\
5. Wine Quality (Red and White Vino Verde): \href{https://archive.ics.uci.edu/dataset/186/wine+quality}{\underline{Wine Quality}} 

\section{How many data points or instances in each dataset?}
1. The adult income dataset has \textbf{48,842} instances, split across train and test data \\
2. The air quality dataset has \textbf{9,358} instances \\
3. The electrical energy prediction dataset has \textbf{71,225} instances \\
4. The heart disease dataset has \textbf{303} instances \\
5. The wine quality dataset is \textbf{1,599 red wine} and \textbf{4,898 white wine} instances

\section{How many features in each dataset?}
1. The adult income dataset has \textbf{14} features (attributes per person) \\
2. The air quality dataset has \textbf{15} features (readings from sensors) \\
3. The electrical energy prediction dataset has \textbf{1} feature (time series) \\
4. The heart disease dataset has \textbf{13} features (patient attributes) \\
5. The wine quality dataset has \textbf{11} features (chemical test information)

\section{What type of person might have collected this data?}
1. The adult income dataset would be collected by a \textbf{census bureau} \\
2. The air quality dataset would have been collected by a \textbf{government or environmental agency} \\
3. The electrical energy prediction dataset would have been collected by an \textbf{energy department} \\
4. The heart disease dataset would have been collected by a \textbf{medical institute} \\
5. The wine quality dataset would have been collected by a \textbf{wine producer}

\section{Why do I find each of the datasets interesting?}
1. Adult Income: It would be interesting to see how well demographics can predict income \\
2. Air Quality: I'm interested in understanding how certain attributes can determine extent of air pollution \\
3. Electrical Energy: I find time series analysis quite interesting, and this seemed quite realistic \\
4. Heart Disease: Understanding the effect of someone's attributes on heart disease can have very positive health impact \\
5. Wine Quality: \textbf{I love wine!}

\section{What are some deeper insights the datasets might reveal?}
1. Adult Income: If demographics can predict income, governments can guide policy to support their citizens \\
2. Air Quality: The data can be used to plan cities in a way which reduces air pollution without other changes \\
3. Electrical Energy: The data can show insights into how the turbine reacts to changes in control voltage \\
4. Heart Disease: If there is a link between certain factors and heart disease, we could have early prediction and reduce medical costs for individuals and governments \\
5. Wine Quality: A winemaker can derive an exact recipe to provide the highest quality wine

\section{What are some ethical or representative issues with the data?}
1. Adult Income: This data is from 1994 and may no longer represent the world today. While the features can help us with predicting income, it can be difficult to attribute rationale if there is a deeper layer of inherent socio-economic bias. \\
2. Air Quality: This is only from one location and might not be representative of other places across the globe. \\
3. Electrical Energy: The data might not generalise to other turbine types or environments. \\
4. Heart Disease: The data might not generalise to the broader demographic and could also lead to privacy issues given medical data. \\
5. Wine Quality: The ratings for each of the wine is subjective to an individual's preferences - it's difficult to remove human bias from the target variable.



\end{document}
